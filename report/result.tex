\chapter{Results and discussion}

\section{Gram-Schmidt orthogonalization using the \textit{for} directive}

\begin{figure}[ht]
  \begin{center}
         \resizebox{160mm}{!}{\includegraphics{pic/graph_gram.eps}}
  \end{center}
  \caption{Gram-Schmidt speedup}
  \label{fig:gram_speedup}
\end{figure} 

As usual with very small matrices there is absolutely no speedup. The first speedup is obtained with a matrix of $300\times 300$ elements but it decreases immediately below 1. Matrices from $600\times 600$ to $1000\times 1000$ elements follow the same pattern with a big speedup for a low number of threads (8) and a decrease just after. The $1500\times 1500$ elements matrix has an odd speedup with 64 threads.

Finally, the $2000\times 2000$ matrix has the max speedup for 16 threads and slowly decreases just after but stays on top of the other speedups. This is because the ratio data/overhead is better for the same number of threads, so there is relatively less overhead.

\section{Gram-Schmidt orthogonalization using the \textit{task} directive}

\begin{figure}[ht]
  \begin{center}
         \resizebox{160mm}{!}{\includegraphics{pic/graph_task.eps}}
  \end{center}
  \caption{Gram-Schmidt speedup using tasks}
  \label{fig:gram_speedup_task}
\end{figure} 

\begin{figure}[ht]
  \begin{center}
         \resizebox{160mm}{!}{\includegraphics{pic/graph_task_2.eps}}
  \end{center}
  \caption{Gram-Schmidt speedup using tasks}
  \label{fig:gram_speedup_task_2}
\end{figure} 


