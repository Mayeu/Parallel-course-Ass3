\chapter{Solution method}

\section{The code}

The program implementing the Gram-Schmidt's algorithm, and used in both parts of the assignment, is a direct translation of the given pseudo code. The program takes two arguments, the length of a vector et their number.\\

The code for each part can be found in the appendixes.

\section{Parallelization using the \textit{for} directive}

In this part of the assignment, we parallelized the $j$-loop using the \textit{for} directive. The code can be found in the appendix \ref{for}.

%For the direct parallelization we parallelize the for: \verb*for j from i+1 in col*.  We can not parallelize the main loop since we need the value of q in $i-1$.

%In the task parallelization we use \verb+#pragma omp single+ before the main loop. And we add the task directive \verb+ #pragma omp task private(j, sigma, k, temp_norm)+ in the \verb+for i in column+ loop.

We chose to parallelize that loop because it is the only one to be perfectly parallel. Indeed, if we consider $i$ as being a constant, we can see that each iteration in the $j$-loop can be made alone. This is not the case with the outermost loop, or $i$-loop. Indeed, the $j$-loop uses data that were computed in a previous iteration of the $i$-loop. Therefore, each iteration of this loop needs to be done before the next can begin. It is then impossible to parallelize it.\\ 

\section{Parallelization using \textit{task} directive}

This part has been solved using the \textit{parallel}, \textit{single} and \textit{task} directives. The code can be found in the appendix \ref{task}.\\

Like we did earlier, the threads are spawned for each iteration of the $i$-loop, thanks to the \textit{parallel} directive this time. Then, we use the text{single} directive to make sure only one thread (the master thread) will create the tasks. The other threads are then scheduled to execute the newly created tasks.\\ 

The task is defined as being the content of the innermost loop, the same we parallelized in the previous part. \\
